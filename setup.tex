% !TEX root = thesis.tex

% Declare old font to make .sty compatible
\DeclareOldFontCommand{\tt}{\normalfont\ttfamily}{\mathtt}

% Unicode chars not set up for the use with LaTeX
\DeclareUnicodeCharacter{251C}{\mbox{\kern.23em
  \vrule height2.4exdepth1exwidth.4pt\vrule height2.4ptdepth-1.8ptwidth.23em}}
\DeclareUnicodeCharacter{2500}{\mbox{\vrule height2.4ptdepth-1.8ptwidth.5em}}
\DeclareUnicodeCharacter{2502}{\mbox{\kern.23em \vrule height2.4exdepth1exwidth.4pt}}
\DeclareUnicodeCharacter{2514}{\mbox{\kern.23em \vrule height2.4exdepth-1.8ptwidth.4pt\vrule height2.4ptdepth-1.8ptwidth.23em}}

\usepackage{
  siunitx,
  todonotes,
  booktabs,
  tikz,
  pgfplots
}
\pgfplotsset{compat=newest}
\usetikzlibrary{patterns}

% important for Bayesian Network graphics
\newcolumntype{M}[1]{D{.}{.}{1.#1}}
\usepackage{dcolumn}

% important for using unicode characters in verbatim-environments
\usepackage{alltt}

% align-environment
\usepackage{amsmath}

% code highlighting
\usepackage[outputdir=dist]{minted}
\usemintedstyle{friendly}

\colorlet{haskell}{blue!50!black}
\colorlet{curry}{orange!70!white}
\colorlet{probLog}{violet!70!white}
\colorlet{webPPL}{blue!70!black}

% redefine subsubsections and paragraphs
\usepackage{titlesec}
\titleformat{\subsubsection}{}{}{}{\bfseries}

% for floating verbatim-environment
\usepackage{fancyvrb}

% **************************************************
% Files' Character Encoding
% **************************************************
\PassOptionsToPackage{utf8}{inputenc}
\usepackage{inputenc}


% **************************************************
% Information and Commands for Reuse
% **************************************************
\newcommand{\thesisTitle}{Effectful Programming in
Declarative Languages with an Emphasis on Non-Determinism:
Applications and Formal Reasoning}
\newcommand{\thesisName}{Sandra Dylus}
\newcommand{\thesisSubject}{PhD Thesis}
\newcommand{\thesisDate}{July, 2019}
\newcommand{\thesisVersion}{First Draft}

\newcommand{\thesisFirstReviewer}{Prof. Dr. Michael Hanus}
\newcommand{\thesisFirstReviewerUniversity}{Kiel University}
\newcommand{\thesisFirstReviewerDepartment}{Department of Programming
Languages and Compiler Construction}

\newcommand{\thesisSecondReviewer}{John Doe}
\newcommand{\thesisSecondReviewerUniversity}{Very Cool University}
\newcommand{\thesisSecondReviewerDepartment}{Department of Weird Stuff}

\newcommand{\thesisFirstSupervisor}{\thesisFirstReviewer}
\newcommand{\thesisSecondSupervisor}{John Smith}

\newcommand{\thesisUniversity}{Kiel University}
\newcommand{\thesisUniversityDepartment}{Department of Programming
Languages and Compiler Construction}
\newcommand{\thesisUniversityInstitute}{Institute for Computer Science}
\newcommand{\thesisUniversityGroup}{Faculty of Engineering}


% **************************************************
% Debug LaTeX Information
% **************************************************
%\listfiles


% **************************************************
% Load and Configure Packages
% **************************************************
\usepackage[ngerman,english]{babel} % babel system, adjust the language of the content
\PassOptionsToPackage{% setup clean thesis style
    figuresep=colon,%
    sansserif=false,%
    hangfigurecaption=false,%
    hangsection=true,%
    hangsubsection=true,%
    colorize=full,%
    colortheme=bluemagenta,%
    bibsys=bibtex,%
    bibfile=thesis,%
    bibstyle=authoryear,%
    wrapfooter=false,%
}{cleanthesis}
\usepackage{cleanthesis}

\hypersetup{% setup the hyperref-package options
    pdftitle={\thesisTitle},    %   - title (PDF meta)
    pdfsubject={\thesisSubject},%   - subject (PDF meta)
    pdfauthor={\thesisName},    %   - author (PDF meta)
    plainpages=false,           %   -
    colorlinks=false,           %   - colorize links?
    pdfborder={0 0 0},          %   -
    breaklinks=true,            %   - allow line break inside links
    bookmarksnumbered=true,     %
    bookmarksopen=true          %
  }

% minted-related definitions
\newminted[curry]{haskell}{style=friendly}
\newmintinline[cyinl]{haskell}{style=friendly}
\newmintinline[hinl]{haskell}{}
\newmintinline[cinl]{coq}{}

\usemintedstyle[haskell]{automn}
\usemintedstyle[coq]{tango}

\def\commentbegin{\quad\{\ }
\def\commentend{\}}

\newenvironment{excursus}[1]
{\vspace{0.5cm}
\hrule
\vspace{0.3cm}
\paragraph{Excursus: #1}}
{\vspace{0.3cm}
\hrule
\vspace{0.5cm}}