\cleanchapterquote{There is no deep theoretical reason why this
  program should be rejected [...].}{Adam Chlipala}{(Associate
  Professor of Computer Science)}

This chapter moves the focus from applications using effects like
non-determinism to an approach to model effectful programs in order to
apply equational reasoning.
Equational reasoning is a popular and common tool to prove properties
about functional programs and, thus, prominently used by Haskell
practitioners. \todo{cite a lot}
In particular, \citet{danielsson2006fast} argue that reasoning about
total programs only and expecting the result to carry over to partial
programs is \textit{morally correct}.
Nevertheless, if we are interested in proving properties about partial
or other effectful programs more explicitly, we cannot apply
equational reasoning as it is and need a model for effectful programs.
In the remainder of this chapter we present an approach to model
effectful non-strict functional programs in the proof assistant Coq
and prove exemplary properties about common functions used in
functional languages.
All these properties hold in total languages as well and are, thus,
not new insights.
The main insight of our approach is that a lot of these properties
carry over to partial languages in the sense that these properties
still hold if partial values are at play.
Our approach allows to reason not only about a concrete
effect like partiality or non-determinism, but for a whole class of effects; the
interesting outcome of the this approach is that one proof is
enough to prove a proposition for a whole class of effects.
In the following we will talk about this insight as being able to
prove properties in an effect-generic fashion.

After discussing the general idea for a framework to model effectful
non-strict functional programs, we emphasise on one effect occurring
in Haskell: partiality.
We present a case study that proves properties of common Haskell list
functions.
The transformation from Haskell to Coq is an intuitive manual
process that we introduce as well.

Last but not least we give an outlook on first ideas to model Curry
programs.
The underlying effect of Curry is non-determinism with the special
treatment with respect to sharing.
As we have seen in previous chapters, the combination of
non-determinism and non-strictness lead Curry to chose a
call-time-choice semantics, where variables declare values and not computations.
That is, our idea tackles the problem of modelling a language where
sharing is not an optimisation with respect to evaluation only, but
can be observed.

In summary, this chapter makes the following contributions.

\begin{itemize}
\item We present a generic model of effectful non-strict functional
  programs in the proof assistant Coq.
\item The modelling in Coq is particular interesting since the
  standard implementation of effectful data structures does not please
  the termination checker.
\item We take the reader on an introductory tour of Coq's intricacies in
order to define a valid representation of effectful programs.
\item We argue that our approach is especially interesting for proving
  properties for a variety of effects once-at-for-all instead of
  individually for each new effect of interest.
\item The first ideas on modelling Curry's call-time choice semantics
  gives insights about modelling a language with sharing.
\end{itemize}
 
\section{A Generic Model For Effectful Non-Strict Programs}

In this section we will present our general approach to model
effectful non-strict programs in a dependently-typed language like
Coq.
We illustrate why previous work is not applicable anymore or does not
tackle the same problem as we do.
Since the reasons for the inapplicability of the previous work is due
to rather technical reasons of dependently-typed languages and the
general idea of propositions as types \citep{wadler2015propositions},
we give a detailed explanation to illustrate which steps are necessary
to model our approach in Coq.
That is, this section tries to explain the obstacles and their
solutions to a beginner-friendly Coq audience.

\subsection{Representation of Partial Programs}
The first obstacle that arises when we model Haskell programs in Coq
is the fact that Coq is a total language and most Haskell programs are
not.
That is, we cannot translate partial Haskell programs into Coq as they
are, we need to represent the partial parts of the programs more
explicitly.

Let us take a look at an example.
One of the first partial function that comes to mind is the \hinl{head} function on lists.
In case of the empty list the function is not defined as there is not element to return.

\begin{minted}{haskell}
head :: [a] -> a
head (x:_) = x
\end{minted}

\noindent 
Due to the missing case for the empty list, the above code is
equivalent to the following implementation that uses a pattern for the
empty list to explicitly yield \hinl{undefined}.

\begin{minted}{haskell}
head :: [a] -> a
head []    = undefined
head (x:_) = x
\end{minted}

In a total language like Coq the corresponding head functions usually
takes an additional argument to take care of the undefined behaviour
in case of the empty list.
The following implementation is part of Coq's predefined library for
lists.

\begin{minted}{coq}
Definition hd (A : Type) (default : A) (l : list A) : A = 
  match l with
  | nil => default
  | cons x _ => x
  end.
\end{minted}

Alternatively, we can indicate the potentially undefined behaviour by
changing the result type to \cinl{option A}.
In case of a non-empty list we wrap the head element into the
\cinl{Some}-constructor, and use \cinl{None} as
resulting value for an empty list.
This alternative implementation is predefined under the name
\cinl{head}.

\begin{minted}{coq}
Definition head (A : Type) (l : list A) : option A  = 
  match l with
  | nil => None
  | x :: _ => Some x
  end.
\end{minted}

The latter version suites our problem a bit better than the first:
there is not always a good candidate in scope that can be used as
default value, the latter version fits a general setting better.
However, the latter version is not compositional.
Consider the following example that uses the result of
\cinl{head} as an argument.

\begin{minted}{coq}
Fail Definition exampleList (A : Type) : list A :=
  cons (head nil) nil.
\end{minted}

We cannot use the result as an element of the newly constructed list,
we get a type error.

\begin{quote}
\emph{The term "cons (head nil) nil" has type "list (option ?A)" while it is expected to have type "list A".}
\end{quote}

The list constructor \cinl{cons} expects the head element
of the list to match its type with all the elements of the remaining
list.
Since we want to construct a list of type \cinl{list A},
the head element needs to be of type \cinl{A};
the function \cinl{head}, however, yields a value of type
\cinl{option A}.

As Haskell is a non-strict language, it is crucial that we allow
partial values at top-level as well as within the components of data
structures, that is, within arguments of constructors in order to
model Haskell's partiality correctly.
Due to non-strictness, it is possible that an undefined value is never
demanded during evaluation.
The Haskell expression \hinl{head (1 : undefined)}, for example, does
not trigger the \hinl{undefined} value in the tail of the list,
because \hinl{head} only demands the first list constructor \hinl{:}
and ignores the second argument.
Instead of reusing \cinl{option}, we define a new data type
\cinl{partial} that captures the notion that a value can be undefined
or defined.

\begin{minted}{coq}
Inductive partial (A : Type) :=
| undefined : partial A
| defined   :  A -> partial A.
\end{minted}

In case of the lists we need to adapt the definition of the data type
itself by lifting all arguments to allow partial values as well.

\begin{minted}{coq}
Inductive List (A : Type) :=
| nil : List A
| cons : partial A -> partial (List A) -> List A.

Definition Nil (A : Type) : partial (List A) :=
  defined nil.
Definition Cons (A : Type) (ox : partial A) (oxs : partial (List A)) 
  : partial (List A) :=
  defined (cons ox oxs).
\end{minted}

The data type \cinl{List} represents a Haskell list in Coq
as all constructor arguments may be partial.
When defining functions on lists, we need to consider that the overall
list may already be partial.
Hence, we define smart constructors \cinl{Nil} and
\cinl{Cons} that wraps the list in the additional
\cinl{defined}-constructor to indicate that the top-level
expression is defined.
The definition of  functions like \cinl{head} adapt to the settings as
follows: all argument as well as result types need to be lifted and
the implementation needs to handle the partial values accordingly.

\begin{minted}{coq}
Definition head (A : Type) (oxs : partial (List A)) : partial A :=
  match oxs with
  | undefined  => undefined
  | defined xs => match xs with
                 | nil       => undefined
                 | cons ox _ => ox
                 end
  end.
\end{minted}

We cannot pattern match on the list directly as in the original
defintion.
In order to access the list, we need to take a look at the
\cinl{partial}-layer first.
If the input list is already \cinl{undefined}, the
overall result is \cinl{undefined} as well.
If the input list \cinl{defined}, we can reimplement the original
behaviour of the Haskell function, that is, yielding the first element
in case of a non-empty list and yielding \cinl{undefined} in case of
an empty list.

Let us take a look at another list function in Haskell and its
counterpart in our model for Coq: concatenation of two lists.

\begin{minted}{haskell}
(++) :: [a] -> [a] -> [a]
(++) []       ys = ys
(++) (x : xs) ys = x : ((++) xs ys)
\end{minted}

The concatenation function on lists is a recursive and total function
and can, thus, be translated to Coq as it is.
However, since we want to mix partial and total functions and, thus,
increase the compositionality of both possible effects, we translate
the append function using the same scheme as above.

\begin{minted}{coq}
Fail Fixpoint append (A : Type) (oxs oys : partial (List A)) : partial (List A) :=
  match oxs with
  | undefined  => undefined
  | defined xs => match xs with
                 | nil         => oys
                 | cons oz ozs => Cons oz (append ozs oys)
                 end
  end.
\end{minted}

The function \cinl{append} is recursive and, as far as we
are concerned, the first argument is structurally decreasing in each
recursive call.
However, Coq does not accept the above definition and rejects its with
the following message.

\begin{quote}
\emph{The command has indeed failed with message: Cannot guess decreasing
argument of fix.
Recursive call to append has principal argument equal to "ozs" instead of a subterm of "oxs".}
\end{quote}

What is going on here?
Coq cannot retrace that the value \cinl{ozs} is actually a
subterm of \cinl{oxs}; without that connection the
recursive call on \cinl{ozs} is not allowed.

The problem is our lifted data type definition of
\cinl{List}.
Due to the lifting, the \cinl{List} becomes a so-called
\emph{nested inductive} type, because the type we want to define
appears nested in another inductive type.
In this case the other inductive type is \cinl{partial}.

We can fix the problem by splitting the function into a non-recursive
and a recursive part.
The recursive part is working only on \cinl{List A} and
produces an \cinl{partial (List A)}, while the non-recursive
function expects and yields a value of type \cinl{partial
  (List A)} and mainly calls the recursive one after unwrapping the
\cinl{partial}-layer.

\begin{minted}{coq}
Definition append (A : Type) (oxs oys : partial (List A)) : partial (List A) :=
  match oxs with
  | undefined  => undefined
  | defined xs =>
    let fix append' xs oys :=
        match xs with
        | nil         => oys
        | cons oz ozs => Cons oz (match ozs with
                                  | undefined => undefined
                                  | defined zs => append' zs oys
                                  end)
        end
    in append' xs oys
  end.
\end{minted}

The recursive function is the local function
\cinl{append'}.
In order to make the recursive call we need to unpack an additional
\cinl{option}-layer of the variable \cinl{ozs}.
Since we are now reconstructing the nested inductive structure using a
nested recursion, Coq accepts the definition.

In the remaining of this chapter we will define the two functions
necessary to translate a recursive Haskell function on top-level to
reuse the definitions in proofs.
The recursive function that works directly on the nested inductive
type will be suffixed with prime like for the local function definition.
That is, the following code shows how the above can be split into two
top-level functions.

\begin{minted}{coq}
Fixpoint append' (A : Type) (xs : List A) (oys : partial (List A)) : partial (List A) :=
  match xs with
  | nil         => oys
  | cons oz ozs => Cons oz (match ozs with
                            | undefined    => undefined
                            | defined zs => append' zs oys
                            end)
  end.

Definition append (A : Type) (oxs oys : partial (List A)) : partial (List A) :=
  match oxs with
  | undefined  => undefined
  | defined xs => append' xs oys
  end
\end{minted}

\subsection{Generalisation Attempt}

Up to now, we have seen an encoding of Haskell programs in Coq that
lifts all Haskell values to \cinl{partial} values.
The overall goal is to not only reason about Haskell's partiality
effect but a whole class of effects.
That is, we do not want to restrict our model in Coq to one effect
only, but generalise it to arbitrary effects.

The first idea that comes to mind to generalise the above encoding is to change the
concrete type \cinl{partial} by a type parameter.
Hence, we can regain the above encoding by instantiating the type
parameter with \cinl{partial}.
Consider the following parametrised definition of \cinl{List}.

\begin{minted}{coq}
Fail Inductive List (M : Type -> Type) (A : Type) :=
| nil : List M A
| cons : M A -> M (List M A) -> List M A.
\end{minted}

Similar to \cinl{partial} that mimics the Maybe monad that we know
from Haskell, we will be mostly interested in other monadic
instantiations.
Thus, we name the type parameter \cinl{M} to indicate that it is a
placeholder for a monadic type.
When we, however, try to compile the above definition, we get the following error message.

\begin{quote}
\emph{The command has indeed failed with message:
Non strictly positive occurrence of "List" in "M A $\rightarrow$ M (List M A) $\rightarrow$ List M A".}
\end{quote}

Coq does not allow to define a type definition like \cinl{List}
because of its \emph{strict positivity requirement}.
What is going on this time?

As a first step to understand what is going on here, we will reduce
the above definition to the simplest definition that still triggers
the same error message.

\begin{minted}{coq}
Fail Inductive NonStrictlyPos :=
| con : (NonStrictlyPos -> nat) -> NonStrictlyPos.
\end{minted}

The data type \cinl{NonStrictlyPos} has no type parameters and only
one unary constructor \cinl{con}.
The argument of the constructor \cinl{con} is a function from
\cinl{NonStrictlyPos} to \cinl{nat}.
The domain of this function is the crucial part of the definition and
the origin of the error message.
As so often when using Coq, \citeauthor{chlipala2011certified}'s book is a
great resource for typical obstacles beginners may face when using Coq
for the first time; he has the following intuitive description for the
strict positivity requirement.

\begin{quote}
\emph{We have run afoul of the strict positivity requirement for
  inductive definitions, which says that the type being defined may
  not occur to the left of an arrow in the type of a constructor
  argument.}
\end{quote}

Inductive type definitions obey the strict positivity requirement, if
for all constructors its recursive occurrences only occur strictly
positive in all argument types.
That is, in case of \cinl{NonStrictlyPos} we need to inspect the types
of the arguments of its only constructor \cinl{con}.
The only argument of \cinl{con} is the function type
\cinl{NonStrictlyPos -> nat}.

\vspace{0.5cm}
\hrule
\vspace{0.3cm}

\paragraph{Excursus: Positive and Negative Occurrences}
What is a positive occurence that \cinl{List} seems to violate giving
the error message Coq produces when we try to compile the data type
definition?
Taking the quote of Chlipala to a more formal ground, we can describe
a strictly positive occurrence of a type as follows: a type $\tau$
occurs strictly positively in a type $\tau_1\rightarrow \dots
\rightarrow \tau_n \rightarrow tau$, if and only if $\tau$ does not
occur in any of the types $\tau_i$ with $i$ ranging from $1$ to $n$.
Although this description might answer the question we raised to some
extent, but it also opens new ones: how do we distinguish between
\emph{positive} and \emph{strictly positive} occurrences; can an
occurrence be \emph{negative} as well?

For that matter, let us consider an arbitrary type $\tau = t_1
\rightarrow \dots \rightarrow t_n$, where all $t_i$ are types as well.
The type arguments $t_1$ to $t_n$ appear in positive or negative
positions.
A type argument $t_i$ has a negative position if it appears to the
left of an odd number of type arrows.
That is, starting from an inner position, each type arrow flips a
negative position to a positive and vice versa.
In particular, a type argument $t_i$ has a positions position if it
appears to the left of an even number of type arrows or no type arrows
at all.
We can define the set of negative and positive positions, $Pos^{+}$
and $Pos^{-}$, respectively, of a type $\tau$ inductively given a set of
inductive types $I$ as described by
\citet{blanqui2002inductivedatatype}.

\begin{alignat*}{2}
  &Pos^{+} (\tau \in I) &&= \epsilon\\
  &Pos^{-} (\tau \in I) &&= \{\}\\
  &Pos^{+} (t \rightarrow \tau) &&= 1 \cdot Pos^{-} (t) \cup 2 \cdot Pos^{+} (\tau)\\
  &Pos^{-} (t \rightarrow \tau) &&= 1 \cdot Pos^{+} (t) \cup 2 \cdot Pos^{-} (\tau)
\end{alignat*}

Based on this definition, we can specify that a type $t$ occurs
positively in $\tau$ if the set $Pos^{+} (\tau)$ contains every occurrence of $\tau$ in
$t$.
Moreover, a type $t$ occurs strictly positively in

\[
\tau = t_1 \rightarrow \dots \rightarrow t_n \rightarrow t
\]

if $t$ occurs in no $t_i$.

Consider the following examples of possible types in Coq, where
\cinl{A} is a type variable.

\begin{enumerate}
\item[(1)] \cinl{A -> nat}
\item[(2)] \cinl{nat -> A}
\item[(3)] \cinl{(A -> nat) -> nat}
\item[(4)] \cinl{((A -> nat) -> nat) -> nat}
\end{enumerate}

The type \cinl{A} occurs positively in (2) and (3), and negatively in
(1) and (4).
The occurrence in (2) is also strictly positive.

In dependently typed languages, the strict positivity
requirement plays an important role for the definition of inductive
data types.
An inductive type t obeys the strict positivity requirement if the
recursive occurrences of t are strictly positively in the types of all
the arguments of its constructors.\footnote{The definition by
  \citeauthor{blanqui2002inductivedatatype} also takes care of mutual
  inductive types where the recursive occurrence is only implicit.}

Consider once again the following exemplary type definitions in Coq.

\begin{enumerate}
\item[(1)] \cinl{Inductive T := con : T -> T -> T}
\item[(2)] \cinl{Inductive T := con : (T -> T) -> T}
\item[(3)] \cinl{Inductive T := con : (nat -> T) -> T}
\item[(4)] \cinl{Inductive T := con : (nat -> T) -> (T -> T) -> T}
\end{enumerate}

The inductive type \cinl{T} obeys the strict positivity requirement in
(1) and (3), and violates it in (2) and (4).
In (1) we need to check the following arguments of the constructors
\cinl{con}: \cinl{T} and \cinl{T}; both types are strictly positive.
The argument in question for (3) is \cinl{nat -> T}, where \cinl{T}
does not occur to the left of an arrow, thus, the type is strictly
positive as well.
In (2) we need to inspect the type \cinl{T -> T}; here, \cinl{T}
occurs to the left of an arrow, thus, in a negative position.
The overall definition, then, does not fulfil the requirement.
The inductive type in (4) is a combination of (2) and (3), thus, the
type of the first argument of \cinl{con} fulfils the requirement,
whereas the second arguments violates it again.

\vspace{0.3cm}
\hrule
\vspace{0.5cm}

Let us think again about the definition of \cinl{NonStrictlyPos}.
With the excursus above, we now know that \cinl{NonStrictlyPos}
occurs ``left of an arrow''  in the first argument of the
\cinl{con}-constructor, namely in the function type
\cinl{NonStrictlyPos -> nat}.
Due to this non strictly positive occurence, the overall data type
definition of \cinl{NonStrictlyPos} is non strictly positive as well,
that is, fails the strict positivity requirement for data type
definitions.
In contrast, the following definition is not problematic.

\begin{minted}{coq}
Inductive StrictlyPos :=
| con : StrictlyPos -> (nat -> StrictlyPos) -> StrictlyPos.
\end{minted}

Here, both argument of the constructor \cinl{con} are strictly positive:
the first argument is of type \cinl{StrictlyPos}, thus, trivially
strictly positive; and in the second argument's type \cinl{StrictlyPos}
does not occur to the left, but to the right of an arrow, fulfilling
the positivity requirement as well.

This little excursus explains the origin of the error message Coq
presents us when we try to define data types likes
\cinl{NonStrictlyPos} and \cinl{List}, but two question still remain
open: (1) Why is the strict positivity of data type definition
required in Coq; (2) Why is our definition \cinl{List} non strictly
positive as well, there is not even an arrow in the arguments' type of
\cinl{cons}?

\paragraph{Why does Coq require the strict positivity of data type
  definitions?}

Let us for a moment assume that the definition of
\cinl{NonStrictlyPos} was allowed.
We then define the following function.\footnote{We use a different
  font to distinguish the following hypothetical code from valid Coq code.}

\begin{verbatim}
Definition applyFun (t : NonStrictlyPos) : nat :=
  match t with
  | con f => f t
  end.
\end{verbatim}

The function \cinl{applyFun} takes a value \cinl{t} of type
\cinl{NonStrictlyPos} and applies the function \cinl{f} inside the
argument of the \cinl{con}-constructor to the value \cinl{t} itself.
A problematic example usage of this function is the expression
\cinl{applyFun (con applyFun)}.
Reducing the expression by using the definition of \cinl{applyFun} yields
\cinl{applyFun (con applyFun)} again.
We can apply this reduction infinitely often, thus, we have
constructed an infinite expression that does not terminate.

\begin{verbatim}
  applyFun (con applyFun)
= {- Definition of applyFun -}
  applyFun (con applyFun)
= {- Definition of applyFun -}
  ...
= {- Definition of applyFun -}
  applyFun (con applyFun)
\end{verbatim}

Taking a step back, we observe that the data type
\cinl{NonStrictlyPos} has some resemblance with a specialised instance
of the fix-point combinator \cinl{Mu}.

\begin{minted}{coq}
Fail Inductive Mu A :=
| mu : (Mu A -> A) -> Mu A.
\end{minted}

Note that we cannot define the type \cinl{Mu} in Coq.
The connection to the fix-point combinator explains why Coq needs to
restrict such a definition: we know that all Coq programs need to
terminate and we need to use a special keyword \cinl{Fixpoint} or
\cinl{fix} to indicate recursive function in order to apply
termination checks with respect to recursion.
If we could define a data type like \cinl{Mu}, our definition of
\cinl{applyFun} is an example that shows how we can define
non-terminating programs without using the explicit fix-point
combinator that Coq already provides for recursive constructions.
That is, data types like \cinl{Mu} introduce the capability to express
general recursion and Coq's logic becomes inconsistent.
In order to ensure the consistency of Coq's logic, the strict
positivity requirement needs to be enforced for data type definitions.

\paragraph{Why does \cinl{List} not fulfil the strict positivity requirement?}

Now we know what Coq is nagging about when the error message about a
non strictly positive occurrences for data type definitions appears.
However, it is still not obvious why the definition of \cinl{List}
falls under the same category.
Recall our try to define the data type \cinl{List}.

\begin{minted}{coq}
Fail Inductive List (M : Type -> Type) (A : Type) :=
| nil : List M A
| cons : M A -> M (List M A) -> List M A.
\end{minted}

The type \cinl{List} does not appear to the left of an arrow in the
argument types of its constructor \cinl{cons}, but Coq still rejects
the definition because of the type \cinl{M (List M A)}.
However, Coq is of course on the right track rejecting such a
definition.
While a concrete instantiation using \cinl{partial} instead of the
type parameter \cinl{M} is accepted, since \cinl{List} does not appear
to the left of an arrow, we cannot guarantee that all usages of
\cinl{List} obey this requirement.

For example, consider the following definition that we might use to
instantiate the type parameter with.

\begin{minted}{coq}
Definition Cont R A := (A -> R) -> R.
\end{minted}

The type \cinl{Cont} represents the continuation monad.
Now let us instantiate the type parameter \cinl{M} in the definition
of \cinl{List} with \cinl{Cont}, a concrete monad.
The following type definition \cinl{ListCont} inlines the definition
of \cinl{Cont} in the constructor corresponding \cinl{cons}.

\begin{minted}{coq}
Fail Inductive ListCont R A :=
| nilC  : ListCont R A
| consC : ((A -> R) -> R) -> ((ListCont R A -> R) -> R) -> ListCont R A.
\end{minted}

Due to the inlining, it now becomes apparent that the type we are just
defining, \cinl{ListCont}, appears indeed to the left of an arrow in
one of the type argument of its constructor.
More precisely, \cinl{ListCont} appears to the left of an arrow in the
second argument of the \cinl{consC}-constructor.
Summarising our findings, the type definition \cinl{List} defined
above allows arbitrary type constructor as instances for its type
parameter \cinl{M}.
Since the type parameter is arbitrary, it is not safe to use this
definition for all potential instances of \cinl{M}.
The strict positivity restriction might be violated for a concrete
instation of \cinl{M}, for instance for the concrete instantiation of
\cinl{Cont} as we demonstrated in the definition of \cinl{ListCont}.
Since we cannot guarantee by definition that the data type definition
\cinl{List} is only used with instantiations of \cinl{M} that obey the
strict positivity requirement, Coq rejects the definition.

\subsection{Free Monad and Containers}
\subsection{Generic Properties and Their Proofs}

\section{Partiality as Effect to Model Haskell Programs}
\subsection{Translation of Haskell Programs}
\subsection{Case Study on List-Related Functions}

\section{First Ideas to Model Curry Programs}
\subsection{Non-Determinism as Effect}
\subsection{Sharing as Effect}

\section{Conclusion and Future Work}

\section{Final Remarks}
