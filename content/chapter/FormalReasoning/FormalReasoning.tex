\cleanchapterquote{There is no deep theoretical reason why this
  program should be rejected [...].}{Adam Chlipala}{(Associate
  Professor of Computer Science)}

This chapter moves the focus from applications using effects like
non-determinism to an approach to model effectful programs in order to
apply equational reasoning.
Equational reasoning is a popular and common tool to prove properties
about functional programs and, thus, prominently used by Haskell
practitioners. \todo{cite a lot}
In particular, \citet{danielsson2006fast} argue that reasoning about
total programs only and expecting the result to carry over to partial
programs is \textit{morally correct}.
Nevertheless, if we are interested in proving properties about partial
or other effectful programs more explicitly, we cannot apply
equational reasoning as it is and need a model for effectful programs.
In the remainder of this chapter we present an approach to model
effectful non-strict functional programs in the proof assistant Coq
and prove exemplary properties about common functions used in
functional languages.
All these properties hold in total languages as well and are, thus,
not new insights.
The main insight of our approach is that a lot of these properties
carry over to partial languages in the sense that these properties
still hold if partial values are at play.
Our approach allows to reason not only about a concrete
effect like partiality or non-determinism, but for a whole class of effects; the
interesting outcome of the this approach is that one proof is
enough to prove a proposition for a whole class of effects.
In the following we will talk about this insight as being able to
prove properties in an effect-generic fashion.

After discussing the general idea for a framework to model effectful
non-strict functional programs, we emphasise on one effect occurring
in Haskell: partiality.
We present a case study that proves properties of common Haskell list
functions.
The transformation from Haskell to Coq is an intuitive manual
process that we introduce as well.

Last but not least we give an outlook on first ideas to model Curry
programs.
The underlying effect of Curry is non-determinism with the special
treatment with respect to sharing.
As we have seen in previous chapters, the combination of
non-determinism and non-strictness lead Curry to chose a
call-time-choice semantics, where variables declare values and not computations.
That is, our idea tackles the problem of modelling a language where
sharing is not an optimisation with respect to evaluation only, but
can be observed.

In summary, this chapter makes the following contributions.

\begin{itemize}
\item We present a generic model of effectful non-strict functional
  programs in the proof assistant Coq.
\item The modelling in Coq is particular interesting since the
  standard implementation of effectful data structures does not please
  the termination checker.
\item We take the reader on an introductory tour of Coq's intricacies in
order to define a valid representation of effectful programs.
\item We argue that our approach is especially interesting for proving
  properties for a variety of effects once-at-for-all instead of
  individually for each new effect of interest.
\item The first ideas on modelling Curry's call-time choice semantics
  gives insights about modelling a language with sharing.
\end{itemize}
 
\section{A Generic Model For Effectful Non-Strict Programs}

In this section we will present our general approach to model
effectful non-strict programs in a dependently-typed language like
Coq.
We illustrate why previous work is not applicable anymore or does not
tackle the same problem as we do.
Since the reasons for the inapplicability of the previous work is due
to rather technical reasons of dependently-typed languages and the
general idea of propositions as types \citep{wadler2015propositions},
we give a detailed explanation to illustrate which steps are necessary
to model our approach in Coq.
That is, this section tries to explain the obstacles and their
solutions to a beginner-friendly Coq audience.

\subsection{Representation of Partial Programs}
The first obstacle that arises when we model Haskell programs in Coq
is the fact that Coq is a total language and most Haskell programs are
not.
That is, we cannot translate partial Haskell programs into Coq as they
are, we need to represent the partial parts of the programs more
explicitly.

Let us take a look at an example.
One of the first partial function that comes to mind is the \hinl{head} function on lists.
In case of the empty list the function is not defined as there is not element to return.

\begin{minted}{haskell}
head :: [a] -> a
head (x:_) = x
\end{minted}

\noindent 
Due to the missing case for the empty list, the above code is
equivalent to the following implementation that uses a pattern for the
empty list to explicitly yield \hinl{undefined}.

\begin{minted}{haskell}
head :: [a] -> a
head []    = undefined
head (x:_) = x
\end{minted}

In a total language like Coq the corresponding head functions usually
takes an additional argument to take care of the undefined behaviour
in case of the empty list.
The following implementation is part of Coq's predefined library for
lists.

\begin{minted}{coq}
Definition hd (A : Type) (default : A) (l : list A) : A = 
  match l with
  | nil => default
  | cons x _ => x
  end.
\end{minted}

Alternatively, we can indicate the potentially undefined behaviour by
changing the result type to \cinl{option A}.
In case of a non-empty list we wrap the head element into the
\cinl{Some}-constructor, and use \cinl{None} as
resulting value for an empty list.
This alternative implementation is predefined under the name
\cinl{head}.

\begin{minted}{coq}
Definition head (A : Type) (l : list A) : option A  = 
  match l with
  | nil => None
  | x :: _ => Some x
  end.
\end{minted}

The latter version suites our problem a bit better than the first:
there is not always a good candidate in scope that can be used as
default value, the latter version fits a general setting better.
However, the latter version is not compositional.
Consider the following example that uses the result of
\cinl{head} as an argument.

\begin{minted}{coq}
Fail Definition exampleList (A : Type) : list A :=
  cons (head nil) nil.
\end{minted}

We cannot use the result as an element of the newly constructed list,
we get a type error.

\begin{quote}
The term "cons (head nil) nil" has type "list (option ?A)" while it is expected to have type "list A".
\end{quote}

The list constructor \cinl{cons} expects the head element
of the list to match its type with all the elements of the remaining
list.
Since we want to construct a list of type \cinl{list A},
the head element needs to be of type \cinl{A};
the function \cinl{head}, however, yields a value of type
\cinl{option A}.

In order to model Haskell's partiality correctly, it is crucial that
we allow partial values at top-level as well as within the components
of data structures, that is, within arguments of constructors.
In case of the lists we need to adapt the definition of the data type
itself by lifting all arguments to allow partial values as well.

\begin{minted}{coq}
Inductive List (A : Type) :=
| nil_ : List A
| cons_ : option A -> option (List A) -> List A.

Definition Nil (A : Type) : option (List A) :=
  Some nil_.
Definition Cons (A : Type) (ox : option A) (oxs : option (List A)) 
  : option (List A) :=
  Some (cons_ ox oxs).
\end{minted}

The data type \cinl{List} represents a Haskell list in Coq
as all constructor arguments may be partial.
When defining functions on lists, we need to consider that the overall
list may already be partial.
Hence, we define smart constructors \cinl{Nil} and
\cinl{Cons} that wraps the list in the additional
\cinl{Some}-constructor to indicate that the top-level
expression is defined.
The definition of  functions like \cinl{head} adapt to the settings as
follows: all argument as well as result types need to be lifted and
the implementation needs to handle the partial values accordingly.

\begin{minted}{coq}
Definition head (A : Type) (oxs : option (List A)) : option A :=
  match oxs with
  | None    => None
  | Some xs => match xs with
              | nil_       => None
              | cons_ ox _ => ox
              end
  end.
\end{minted}

We cannot pattern match on the list directly as in the original
defintion.
In order to access the list, we need to take a look at the
\cinl{option}-layer first.
If the input list is already undefined (\cinl{None}), the
overall result is \cinl{None} as well.
If the input list is defined (\cinl{Some}), we can
reimplement the original behaviour of the Haskell function, that is, yielding
the first element in case of a non-empty list and yielding
\cinl{None} as representative for undefined values in case
of an empty list.

Let us take a look at another list function in Haskell and its
counterpart in our model for Coq: concatenation of two lists.

\begin{minted}{haskell}
(++) :: [a] -> [a] -> [a]
[]     ++ ys = ys
x : xs ++ ys = x : (xs ++ ys)
\end{minted}

The concatenation function on lists is a recursive and total function
and can, thus, be translated to Coq as it is.
However, since we want to mix partial and total functions and, thus,
increase the compositionality of both possible effects, we translate
the append function using the same scheme as above.

\begin{minted}{coq}
Fail Fixpoint append (A : Type) (oxs oys : option (List A)) : option (List A) :=
  match oxs with
  | None    => None
  | Some xs => match xs with
              | nil_         => oys
              | cons_ oz ozs => Cons oz (append ozs oys)
              end
  end.
\end{minted}

The function \cinl{append} is recursive and, as far as we
are concerned, the first argument is structurally decreasing in each
recursive call.
However, Coq does not accept the above definition and rejects its with
the following message.

\begin{quote}
The command has indeed failed with message: Cannot guess decreasing
argument of fix.
Recursive call to append has principal argument equal to "ozs" instead of a subterm of "oxs".
\end{quote}

What is going on here?
Coq cannot retrace that the value \cinl{ozs} is actually a
subterm of \cinl{oxs}; without that connection the
recursive call on \cinl{ozs} is not allowed.

The problem is our lifted data type definition of
\cinl{List}.
Due to the lifting, the \cinl{List} becomes a so-called
\emph{nested inductive} type, because the type we want to define
appears nested in another inductive type.
In this case the other inductive type is \cinl{option}.

We can fix the problem by splitting the function into a non-recursive
and a recursive part.
The recursive part is working only on \cinl{List A} and
produces an \cinl{option (List A)}, while the non-recursive
function expects and yields a value of type \cinl{option
  (List A)} and mainly calls the recursive one after unwrapping the
\cinl{option}-layer.

\begin{minted}{coq}
Definition append (A : Type) (oxs oys : option (List A)) : option (List A) :=
  match oxs with
  | None    => None
  | Some xs =>
    let fix append' xs oys :=
        match xs with
        | nil_         => oys
        | cons_ oz ozs => Cons oz (match ozs with
                                  | None => None
                                  | Some zs => append' zs oys
                                  end)
        end
    in append' xs oys
  end.
\end{minted}

The recursive function is the local function
\cinl{append'}.
In order to make the recursive call we need to unpack an additional
\cinl{option}-layer of the variable \cinl{ozs}.
Since we are now reconstructing the nested inductive structure using a
nested recursion, Coq accepts the definition.

In the remaining of this chapter we will define the two functions
necessary to translate a recursive Haskell function on top-level to
reuse the definitions in proofs.
The recursive function that works directly on the nested inductive
type will be prefixed with prime like for the local function definition.
That is, the following code shows how the above can be split into two
top-level functions.

\begin{minted}{coq}
Fixpoint append' (A : Type) (xs : List A) (oys : option (List A)) : option (List A) :=
  match xs with
  | nil_         => oys
  | cons_ oz ozs => Cons oz (match ozs with
                            | None    => None
                            | Some zs => append' zs oys
                            end)
  end.

Definition append (A : Type) (oxs oys : option (List A)) : option (List A) :=
  match oxs with
  | None    => None
  | Some xs => append' xs oys
  end.
\end{minted}

\subsection{Generalisation}

\subsection{Free Monad and Containers}
\subsection{Generic Properties and Their Proofs}

\section{Partiality as Effect to Model Haskell Programs}
\subsection{Translation of Haskell Programs}
\subsection{Case Study on List-Related Functions}

\section{First Ideas to Model Curry Programs}
\subsection{Non-Determinism as Effect}
\subsection{Sharing as Effect}

\section{Conclusion and Future Work}

\section{Final Remarks}
