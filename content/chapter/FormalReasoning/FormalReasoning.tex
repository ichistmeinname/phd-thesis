
This chapter moves the focus from applications using effects like
non-determinism to an approach to model effectful programs in order to
apply equational reasoning.
Equational reasoning is a popular and common tool to prove properties
about functional programs and, thus, prominently used by Haskell
practitioners. \todo{cite a lot}
In particular, \citet{danielsson2006fast} argue that reasoning about
total programs only and expecting the result to carry over partial
programs is \textit{morally correct}.
Nevertheless, if we are interested in proving properties about partial
or other effectful programs more explicitly, we cannot apply
equational reasoning as it is and need a model for effectful programs.
In the remainder of this chapter we present an approach to model
effectful non-strict functional programs in the proof assistant Coq
and prove exemplary properties about common functions used in
functional languages.
All these properties hold in total languages as well and are, thus,
not new insights.
The main insight of our approach is that a lot of these properties
carry over to partial languages in the sense that these properties
still hold if partial values are at play.
In the following we will talk about this insight as being able to
prove properties in an effect-generic fashion.

After discussing the general idea for a framework to model effectful
non-strict functional programs, we emphasise on one effect occurring
in Haskell: partiality.
We present a case study that proves properties of common Haskell list
functions.
The transformation from Haskell to Coq is an intuitive manual
process that we introduce as well.

Last but not least we give an outlook on first ideas to model Curry
programs.
The underlying effect of Curry is non-determinism with the special
treatment with respect to sharing.
As we have seen in previous chapters, the combination of
non-determinism and non-strictness lead Curry to chose a
call-time-choice semantics, where variables declare values and not computations.
That is, our idea tackles the problem of modelling a language where
sharing is not an optimisation with respect to evaluation only, but
can be observed.

In summary, this chapter makes the following contributions.

\begin{itemize}
\item We present a generic model of effectful non-strict functional
  programs in the proof assistant Coq.
\item The modelling in Coq is particular interesting since the
  standard implementation of effectful data structures does not please
  the termination checker.
We take the reader on an introductory tour of Coq's intricacies in
order to define a valid representation of effectful programs.
\item more
\end{itemize}
 
\section{A Generic Model For Effectful Non-Strict Programs}
\subsection{Goal and Problem}
\subsection{Free Monad and Containers}
\subsection{Generic Properties and Their Proofs}
\section{Partiality as Effect to Model Haskell Programs}
\subsection{Translation of Haskell Programs}
\subsection{Case Study on List-Related Functions}
\section{First Ideas to Model Curry Programs}
\subsection{Non-Determinism as Effect}
\subsection{Sharing as Effect}
\section{Conclusion and Future Work}
\section{Final Remarks}
