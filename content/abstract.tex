% !TEX root = ../my-thesis.tex
%
\begingroup
\let\cleardoublepage\clearpage

\pdfbookmark[0]{Abstract}{Abstract}
\selectlanguage{english}
\chapter*{Abstract}
\label{sec:abstract}
This thesis investigates effectful declarative programming with an
emphasise on non-determinism as an effect.

On the one hand, we are interested in developing applications using
non-determinism as underlying implementation idea.
We discuss two applications in the functional logic programming
language Curry.
The key idea of these implementations is to exploit the interplay of
non-determinism and non-strictness that Curry employs.

The first application is a rather toy example that investigates
sorting algorithms parametrised over a comparison function.
By applying a non-deterministic predicate to these sorting functions,
we gain a permutation enumeration function.
We compare the implementation in Curry with an implementation in
Haskell using a monadic interface to model non-determinism.

The other application that we discuss in this work is a library for
probabilistic programming.
Instead of modelling distributions as list of event and probability
pairs, we model distributions using Curry's built-in non-determinism.

In both cases we observe that the combination of non-determinism and
non-strictness has advantages over an implementation using lists to
model non-determinism.

On the other hand, we present an idea to apply formal reasoning on
effectful declarative programming languages.
In order to start with simple effects, we focus on modelling a
functional subset first.
That is, the effects of interest are totality and partiality.
We then observe that the general scheme to model these two effects can
be generalised to capture a wide range of effects.
Obviously, the next step is to apply the idea to model
non-determinism.
More precisely, we model the non-determinism of Curry: that is,
non-strict non-determinism with call-time-choice.

At the end, we give an outlook of the combinations of both worlds and
sketch the first ideas to formally reason about properties of the
probabilistic programming library and non-deterministic sorting functions.

\selectlanguage{ngerman}
\chapter*{Zusammenfassung (deutsch)}
\label{sec:abstract-diff}

Diese Arbeit besch\"aftigt sich mit der deklarativen
Programmierung mit Effekten und legt dabei besonderen Fokus auf
Nichtdeterminismus als Effekt.

Einerseits m\"ochten wir Anwendungen entwicklen, dessen
zugrundeliegende Implementierunsidee auf Nichtdeterminismus basiert.
Wir stellen dazu zwei beispielhafte Anwendungen vor, die in der
funktional logischen Programmiersprache Curry implementiert sind.
Die Kernidee dieser Implementierung ist dabei die Kombination von
Nichtstriktheit und Nichtdeterminismus, die Curry unterliegen,
gewinnbringend auszunutzen.

Die erste Anwendung ist letztendlich eher eine nette Spielerei mit
Sortierfunktionen als eine echte Anwendung.
Wir untersuchen Sortierfunktionen die \"uber eine Vergleichsfunktion
parametrisiert sind und wenden diese Funktionen auf ein
nichtdeterministisches Pr\"adikat an.
Dabei entsteht eine Funktion, die Permutationen der Eingabeliste
auflistet.
Wir vergleichen unsere Implementierung in Curry mit einer
Implementierung in Haskell, die den Nichtdeterminismus monadisch
modelliert.

Als zweite Anwendung werden wir \"uber eine Bibliothek zur
probabilistischen Programmierung diskutieren.
Statt der \"ublichen Modellierung von Wahrscheinlichkeitsverteilungen
als Liste von Ereignis-Wahrscheinlichtkeits-Paar modellieren wir diese
Verteilungen mithilfe von Currys eingebauten Nichtdeterminismus.

Beide Implementierung haben dadurch die Kombination von
Nichtdeterminismus und Nichtstriktheit Vorteile geben\"uber
einer Implementierung, die den Nichtdeterminismus durch Listen
repr\"asentiert.

Andererseits m\"ochten wir eine M\"oglichkeit schaffen, \"uber die
Programme, die wir in den effektbehafteten deklarativen
Programmiersprachen entwickelt haben, in einem formalen Rahmen zu
argumentieren.
Dabei fangen wir mit einfachen der Teilmenge der rein funktionalen
Effekten, das hei\ss{}t, wir interessieren uns zun\"achst f\"ur
Totalit\"at und Partialit\"at von Programmen.
Die zugrundeliegende Idee zur Modellierung dieser zwei Effekte kann
dann auch f\"ur weitere Effekte genutzt werden.
Als nat\"urlichen n\"achsten Schritt betrachten wir den Effekte, der
bei der Sprache Curry zus\"atzlich hinzukommt: nicht-strikter
Nichtdeterminismus mit \emph{Call-Time-Choice} Semantik.

Zu guter Letzt geben wir einen Ausblick auf die Verbindung dieser
beiden Interessen und pr\"asentieren erste Ideen zur
Modellierung von Curry, um Eigenschaften der Bibliothek zur
probabilistischen Programmierung sowie nicht-deterministischen
Sortierfunktionen zur formulieren und deren Validit\"at
zu pr\"ufen.

\endgroup