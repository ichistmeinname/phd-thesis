% !TEX root = ../my-thesis.tex
%
\begingroup
\let\cleardoublepage\clearpage

\pdfbookmark[0]{Abstract}{Abstract}
\selectlanguage{english}
\chapter*{Abstract}
\label{sec:abstract}
This thesis investigates effectful declarative programming with an
emphasise on non-determinism as an effect.

On the one hand, we are interested in developing applications using
non-determinism as underlying implementation idea.
One application that we discuss in this work is a library for
probabilistic programming developed in the functional logic language
Curry.
The key idea of the implementation is exploiting the interplay of
non-determinism and non-strictness that Curry employs.
Due to this combination even a naive implementation has advantages
over an implementation using lists to model non-determinism.

On the other hand, we present an idea to apply formal reasoning on
effectful declarative programming languages.
In order to start with simple effects, we focus on modelling a
functional subset first.
That is, the effects of interest are totality and partiality.
We then observe that the general scheme to model these two effects can
be generalised to capture a wide range of effects.
Obviously, the next step is to apply the idea for modelling
non-determinism.
More precisely, we model the non-determinism of Curry: that is,
non-strict non-determinism with call-time-choice.

At the end, we combine both worlds and show first efforts to formally
reason about properties of the probabilistic programming library.

\selectlanguage{ngerman}
\chapter*{Abstrakt (deutsch)}
\label{sec:abstract-diff}

Diese Arbeit besch\"aftigt sich mit der deklarativen
Programmierung mit Effekten und legt dabei besonderen Fokus auf
Nichtdeterminismus als Effekt.

Einerseits m\"ochten wir Anwendungen entwicklen, dessen
zugrundeliegende Implementierunsidee auf Nichtdeterminismus basiert.
Als beispielhafte Anwendung werden wir dabei \"uber eine Bibliothek zur
probabilistischen Programmierung diskutieren, die in der funktional logischen
Programmiersprache implementiert wurde.
Der Kern der Implementierung nutzt dabei die Kombination von
Nichtstriktheit und Nichtdeterminismus, die Curry unterliegen,
gewinnbringend aus.
Die entstandene naive Implementierung hat dadurch Vorteile geben\"uber
einer Implementierung, die den Nichtdeterminismus durch Listen repr\"asentiert.

Andererseits m\"ochten wir eine M\"oglichkeit schaffen, \"uber die
Programme, die wir in den effektbehafteten deklarativen
Programmiersprachen entwickelt haben, in einem formalen Rahmen zu
argumentieren.
Dabei fangen wir mit einfachen der Teilmenge der rein funktionalen
Effekten, das hei\ss{}t, wir interessieren uns zun\"achst f\"ur
Totalit\"at und Partialit\"at von Programmen.
Die zugrundeliegende Idee zur Modellierung dieser zwei Effekte kann
dann auch f\"ur weitere Effekte genutzt werden.
Als nat\"urlichen n\"achsten Schritt betrachten wir den Effekte, der
bei der Sprache Curry zus\"atzlich hinzukommt: nicht-strikter
Nichtdeterminismus mit \emph{Call-Time-Choice} Semantik.

Zu guter Letzt verbinden wir diese beiden Interessen und verwenden die
Modellierung von Curry, um Eigenschaften der Bibliothek zur
probabilistischen Programmierung zur formulieren und deren Validit\"at
zu pr\"ufen.

\endgroup